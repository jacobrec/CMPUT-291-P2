\begin{figure}[!h]
	\centering
	\includegraphics[scale=0.5]{flow_dot.png}
	\caption{Program flow}\label{fig:use_flow}
\end{figure}

The design of the database system is created in such a way to maximize efficiency and 
speed while still being user friendly and working as expected.  When the user enters 
a query formatted as above, the query is parsed then put through an optimizer.  This 
optimizer combines date queries into a single date range.  Note that each query only 
allows up to one date range.  

The optimizer then reorders the queries putting queries 
that are most likely to return no results first.  This ends up with date being the 
first query to be handled, returning all results within the range.  After that, the 
queries regarding email address are handled, then finally keyword queries are handled.  
In a multi-part query the system will take the result of each query and take the 
intersection of the result and that of the previous query.  If at any point this 
intersection results in an empty set, the querying terminates, as additional queries 
will not give any more results and will just waste time / computation.  

When the query is completed it will return the result to the user.  The format of the 
result is determined by the output mode (see User Guide for more information).  If the 
result is blank (ie. no output) that means that the result of the optimizer returned 
an empty set.  Thus your subqueries contained no intersection with one another, and there 
was no values in the database that simultaneously satisfied all subqueries.  On the other 
hand, if the program returns an error \verb|DB_NOTFOUND| that means that one of your 
subqueries failed to return any values at all, meaning there was no values in the database 
that satisfied it once it had been run through the optimizer.  Finally if there was a 
syntax error that means your query was not properly formatted.  See the User Guide for 
more information.  For example, a query \verb|date<2000/01/01 date:2000/12/12| will 
contain no intersections and thus return nothing. 
