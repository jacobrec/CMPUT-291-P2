This application has two main parts to it's operation.  The first is the data preparation stage
and the second is the actual searching and operation of the program.  Navigation of the program
can be done using the keyboard, and it is designed to be used within the terminal.  When 
inputting commands be sure to follow the syntax guide described below. 

\subsection{Starting the Program}
In order to set up the program you should locate the xml file you would wish, then run doProject.sh
with the input parameter as the path ro the xml file.  For example 
\verb|./doProject.sh /path/to/file.xml|.  Once that is done, wait for each of the parts to complete
parsing and indexing the file.  Once that is completed you will be presented with a promt in the 
form of \verb|> |, where you will be able to enter commands as described below.  

\subsection{Program Operation}
This program has multiple parameters and inputs that can be used for searching your data set.

\textbf{Keyword Searching: }
If you enter a word without special characters it will return all emails containing that word
in either the subject or the body.  The word must be more than 2 characters long to return a 
result.  If you would like to be more specific you can use either \verb|subj:| to search only
the subject lines for the given keyword, or use \verb|body:| to search only the body.  

You can also use the wildcard character \verb|%| at the end of a keyword to find any results that 
contain the preceeding character sequence.  This can be combined with \verb|subj:| and 
\verb|body:|, for example: \verb|subj:test%| will return emails that contain "test", "tests", 
"testing", "tester"... or any other word that starts with "test" in the subject.  Note that 
using the wildcard you can perform a query such as \verb|i%|, but it will not return emails 
containing words like "is", "it", or similar, as 2 character words are not indexed (as per the 
spec).  Also, the command \verb|%| is not valid, as there must be at least one preceeding valid 
character.  

\textbf{Email Address Searching: }
If you would like to search based on an email address make sure you know the full email address 
for the person you would like to contact, as wildcards are not permitted in these searches.  
\begin{itemize}
	\item To find emails sent by a particular person, use \verb|from:|
	\item To find emails sent to a particular person, use \verb|to:|
	\item To find emails CC'd to a particular person, use \verb|cc:|
	\item To find emails BCC'd to a particular person, use \verb|bcc:|
\end{itemize}
Followed by the email address of the person in question.  For example \verb|from:person@example.com| 
will return all emails sent by "person@example.com".  

\textbf{Date Searching: }
You are able to search emails based on when they are dated as well.  Dates are formatted as 
\verb|YYYY/MM/DD|, and the command to search dates is \verb|date| followed by one of the following 
operators.  
\begin{itemize}
	\item \verb|:| returns emails matching the date exactly
	\item \verb|<| returns emails sent before the given date
	\item \verb|<=| returns emails sent before or on the given date
	\item \verb|>| returns emails send after the given date
	\item \verb|>=| returns emails send after or on the given date
\end{itemize}
Followed by the date in question.  For example \verb|date>2012/01/02| will return all emails dated 
after January 2nd, 2012.  

\textbf{Combined Queries: }
You are able to enter more than one command on a line in order to combine them.  Each query is 
space separated.  These queries are evaluated, then their results are combined to the intersection 
of all outputs.  This means that it will only return results that every single query on the line 
matched with.  In addition, this means that if a query results is an empty set (no results) a 
combined query containing it will also return nothing.  

This allows you you perform more complex searches.  For example, if you wanted to find all emails 
that are sent by john.arnold@enron.com between 1999/01/01 (exclusive) and 2010/12/31 (inclusive) 
containing any word starting with "confident" in the body, you can use the following query:

\verb|from:john.arnold@enron.com date>1999/01/01 date<=2010/12/31 body:confident%|

\textbf{Mode Switching: }
To switch between output modes use the command \verb|output=| followed by either \verb|full| or 
\verb|brief|.  \verb|output=full| will cause all commands entered to return the XML dump of all 
matching emails, while \verb|output=brief| will only return the id and subject line of the matching 
emails.  The default mode is full, and being in brief mode will not change your ability to search 
the body of emails you are interested in, as it only changes how the output is displayed, not how 
the queries are handled. 

\textbf{Exiting: }
To close the program, enter !exit into the prompt, or use CTRL-D (not recommended).   
